\documentclass{article}
\usepackage{Sweave}

\begin{document}
\Sconcordance{concordance:ARMA22_ens.tex:ARMA22_ens.Rnw:1 11 1 1 2 4 0 1 2 2 1 1 3 2 %
0 3 1 1 3 1 0 2 1 1 3 1 0 1 3 1 0 1 19 18 0 1 6 4 0 1 1 1 3 5 0 1 2 3 1 %
1 3 2 0 1 3 1 0 1 1 1 36 34 0 1 39 37 0 1 43 44 0 1 2 3 1 1 3 2 0 2 1 1 %
3 1 0 2 1 1 3 1 0 2 1 1 10 8 0 1 2 2 1 1 5 3 0 1 2 2 1 1 3 1 0 2 1 1 3 %
1 0 2 1 1 10 8 0 1 2 2 1 1 5 3 0 1 2 2 1 1 9 7 0 1 3 1 0 1 1 1 3 1 0 1 %
1 1 3 1 0 1 3 1 0 2 1 1 9 7 0 1 3 1 0 3 1 1 3 1 0 2 1 1 6 7 0 1 2 3 1 1 %
3 14 0 1 1 9 0 1 2 1 1 1 9 8 0 1 3 1 0 2 1 1 3 1 0 1 3 1 0 1 7 5 0 1 13 %
11 0 1 6 4 0 1 8 9 0 1 2 1 1}


\title{ARMA(2,2) Time Series Modeling and Evaluation}
\author{Your Name}
\date{\today}

\maketitle

\begin{Schunk}
\begin{Sinput}
> knitr::opts_chunk$set(echo = TRUE)
\end{Sinput}
\end{Schunk}

\section{Data Generation and Stan Model Setup}

\begin{Schunk}
\begin{Sinput}
> # Load required libraries
> library(forecast)
> library(rstan)
> library(loo)
> library(ggplot2)
> # Define parameters for data generation
> n_people <- 10
> n_burnins <- 25
> n_timepoints <- 50
> # Initialize list to store time series data for each person
> time_series_list <- list()
> # Generate ARMA(2,2) time series data for each person
> set.seed(123)  # Set random seed for reproducibility
> for (i in 1:n_people) {
+   ar <- c(0.5, -0.3)  # Autoregressive coefficients
+   ma <- c(0.4, -0.2)  # Moving average coefficients
+   mu <- 0             # Mean of the process
+   sigma <- 1          # Standard deviation of the process
+   e <- rnorm(n_burnins + n_timepoints, mean = mu, sd = sigma) # White noise
+   data <- numeric(n_burnins + n_timepoints) # Initialize data vector
+   data[1:2] <- e[1:2]                        # Set initial values
+   
+   # Generate ARMA(2,2) process
+   for (t in 3:(n_burnins + n_timepoints)) {
+     data[t] <- mu + ar[1] * (data[t - 1] - mu) + 
+       ar[2] * (data[t - 2] - mu) + 
+       e[t] + ma[1] * e[t - 1] + ma[2] * e[t - 2] 
+   }
+   
+   # Store time series data after discarding burn-in period
+   time_series_list[[i]] <- data[(n_burnins + 1):(n_burnins + n_timepoints)]
+ }
> # Visualize the generated time series data
> plot(1:n_timepoints, type = 'n', 
+      xlim = c(1, n_timepoints), ylim = range(unlist(time_series_list)), 
+      main = "ARMA(2,2) Time Series for All Individuals", 
+      xlab = "Time", ylab = "Value")
> colors <- rainbow(n_people)  # Create a color palette for each individual
> for (i in 1:n_people) {
+   lines(1:n_timepoints, time_series_list[[i]], col = colors[i], lwd = 1)
+ }
\end{Sinput}
\end{Schunk}


\section{Stan Model Definitions}

\begin{Schunk}
\begin{Sinput}
> # Prepare the data for Stan
> data_list <- list(N_people = n_people, N_timepoints = n_timepoints, y = lapply(time_series_list, function(x) x[1:n_timepoints]))
> # Set rstan options
> options(mc.cores = parallel::detectCores())
> rstan_options(auto_write = TRUE)
> # Define the Stan models
> stan_model_AR1 <- stan_model(model_code = "
+ data {
+   int<lower=1> N_people;
+   int<lower=1> N_timepoints;
+   vector[N_timepoints] y[N_people];
+ }
+ parameters {
+   real phi1;
+   real<lower=0> sigma;
+   real mu;
+ }
+ model {
+   phi1 ~ normal(0, 1);
+   sigma ~ cauchy(0, 2.5);
+   mu ~ normal(0, 10);
+   for (j in 1:N_people) {
+     for (n in 2:N_timepoints) {
+       y[j][n] ~ normal(mu + phi1 * (y[j][n-1] - mu), sigma);
+     }
+   }
+ }
+ generated quantities {
+   vector[N_timepoints] log_lik[N_people];
+   vector[N_timepoints] y_hat[N_people];
+   for (j in 1:N_people) {
+     log_lik[j][1] = 0;
+     y_hat[j][1] = y[j][1];
+     for (n in 2:N_timepoints) {
+       log_lik[j][n] = normal_lpdf(y[j][n] | mu + phi1 * (y[j][n-1] - mu), sigma);
+       y_hat[j][n] = normal_rng(mu + phi1 * (y[j][n-1] - mu), sigma);
+     }
+   }
+ }
+ ")
> stan_model_AR2 <- stan_model(model_code = "
+ data {
+   int<lower=1> N_people;
+   int<lower=1> N_timepoints;
+   vector[N_timepoints] y[N_people];
+ }
+ parameters {
+   real phi1;
+   real phi2;
+   real<lower=0> sigma;
+   real mu;
+ }
+ model {
+   phi1 ~ normal(0, 1);
+   phi2 ~ normal(0, 1);
+   sigma ~ cauchy(0, 2.5);
+   mu ~ normal(0, 10);
+   for (j in 1:N_people) {
+     for (n in 3:N_timepoints) {
+       y[j][n] ~ normal(mu + phi1 * (y[j][n-1] - mu) + phi2 * (y[j][n-2] - mu), sigma);
+     }
+   }
+ }
+ generated quantities {
+   vector[N_timepoints] log_lik[N_people];
+   vector[N_timepoints] y_hat[N_people];
+   for (j in 1:N_people) {
+     log_lik[j][1] = 0;
+     log_lik[j][2] = 0;
+     y_hat[j][1] = y[j][1];
+     y_hat[j][2] = y[j][2];
+     for (n in 3:N_timepoints) {
+       log_lik[j][n] = normal_lpdf(y[j][n] | mu + phi1 * (y[j][n-1] - mu) + phi2 * (y[j][n-2] - mu), sigma);
+       y_hat[j][n] = normal_rng(mu + phi1 * (y[j][n-1] - mu) + phi2 * (y[j][n-2] - mu), sigma);
+     }
+   }
+ }
+ ")
> stan_model_AR3 <- stan_model(model_code = "
+ data {
+   int<lower=1> N_people;
+   int<lower=1> N_timepoints;
+   vector[N_timepoints] y[N_people];
+ }
+ parameters {
+   real phi1;
+   real phi2;
+   real phi3;
+   real<lower=0> sigma;
+   real mu;
+ }
+ model {
+   phi1 ~ normal(0, 1);
+   phi2 ~ normal(0, 1);
+   phi3 ~ normal(0, 1);
+   sigma ~ cauchy(0, 2.5);
+   mu ~ normal(0, 10);
+   for (j in 1:N_people) {
+     for (n in 4:N_timepoints) {
+       y[j][n] ~ normal(mu + phi1 * (y[j][n-1] - mu) + phi2 * (y[j][n-2] - mu) + phi3 * (y[j][n-3] - mu), sigma);
+     }
+   }
+ }
+ generated quantities {
+   vector[N_timepoints] log_lik[N_people];
+   vector[N_timepoints] y_hat[N_people];
+   for (j in 1:N_people) {
+     log_lik[j][1] = 0;
+     log_lik[j][2] = 0;
+     log_lik[j][3] = 0;
+     y_hat[j][1] = y[j][1];
+     y_hat[j][2] = y[j][2];
+     y_hat[j][3] = y[j][3];
+     for (n in 4:N_timepoints) {
+       log_lik[j][n] = normal_lpdf(y[j][n] | mu + phi1 * (y[j][n-1] - mu) + phi2 * (y[j][n-2] - mu) + phi3 * (y[j][n-3] - mu), sigma);
+       y_hat[j][n] = normal_rng(mu + phi1 * (y[j][n-1] - mu) + phi2 * (y[j][n-2] - mu) + phi3 * (y[j][n-3] - mu), sigma);
+     }
+   }
+ }
+ ")
\end{Sinput}
\end{Schunk}


\section{Fitting Stan Models and Calculating ICs}

\begin{Schunk}
\begin{Sinput}
> # Fit the Stan models
> fit_combined_AR1 <- sampling(stan_model_AR1, data = data_list, iter = 2000, chains = 4)
> fit_combined_AR2 <- sampling(stan_model_AR2, data = data_list, iter = 2000, chains = 4)
> fit_combined_AR3 <- sampling(stan_model_AR3, data = data_list, iter = 2000, chains = 4)
> # Extract posterior means of the parameters
> posterior_mean_AR1 <- summary(fit_combined_AR1)$summary
> posterior_mean_AR2 <- summary(fit_combined_AR2)$summary
> posterior_mean_AR3 <- summary(fit_combined_AR3)$summary
> # Extract log-likelihoods
> log_lik_AR1 <- extract_log_lik(fit_combined_AR1, merge_chains = TRUE)
> log_lik_AR2 <- extract_log_lik(fit_combined_AR2, merge_chains = TRUE)
> log_lik_AR3 <- extract_log_lik(fit_combined_AR3, merge_chains = TRUE)
> # Reshape log-likelihood matrix to group by individual and time point
> reshape_log_lik <- function(log_lik, n_people, n_timepoints) {
+   n_samples <- nrow(log_lik)
+   reshaped_log_lik <- array(0, dim = c(n_samples, n_people, n_timepoints))
+   for (i in 1:n_people) {
+     reshaped_log_lik[, i, ] <- log_lik[, ((i-1) * n_timepoints + 1):(i * n_timepoints)]
+   }
+   return(reshaped_log_lik)
+ }
> log_lik_AR1_reshaped <- reshape_log_lik(log_lik_AR1, n_people, n_timepoints)
> log_lik_AR2_reshaped <- reshape_log_lik(log_lik_AR2, n_people, n_timepoints)
> log_lik_AR3_reshaped <- reshape_log_lik(log_lik_AR3, n_people, n_timepoints)
> # Calculate total log-likelihood for each person and chain combination
> total_log_lik <- function(log_lik) {
+   apply(log_lik, c(1, 2), sum)  # Sum log-likelihoods across time points
+ }
> log_lik_AR1_total <- total_log_lik(log_lik_AR1_reshaped)
> log_lik_AR2_total <- total_log_lik(log_lik_AR2_reshaped)
> log_lik_AR3_total <- total_log_lik(log_lik_AR3_reshaped)
> # Calculate WAIC and LOOIC using total log-likelihood
> loo_AR1_total <- loo(log_lik_AR1_total)
> loo_AR2_total <- loo(log_lik_AR2_total)
> loo_AR3_total <- loo(log_lik_AR3_total)
> # Extract posterior samples of y_hat for each model
> y_hat_AR1 <- extract(fit_combined_AR1, pars = "y_hat")$y_hat
> y_hat_AR2 <- extract(fit_combined_AR2, pars = "y_hat")$y_hat
> y_hat_AR3 <- extract(fit_combined_AR3, pars = "y_hat")$y_hat
> # Calculate MSE for each model (corrected)
> calculate_mse <- function(y_hat, y_true, model_order) {
+   start_point <- model_order + 1  # match the start_point with the lags
+   mse_values <- sapply(1:dim(y_hat)[1], function(s) {  # Iterate over MCMC samples
+     mse_per_person <- sapply(1:length(y_true), function(i) mean((y_hat[s, i, ] - y_true[[i]])^2))
+     return(mean(mse_per_person)) # Average MSE across individuals for this sample
+   })
+   return(mean(mse_values)) # Average MSE across all MCMC samples
+ }
> mse_AR1 <- calculate_mse(y_hat_AR1, data_list$y, model_order = 1)
> mse_AR2 <- calculate_mse(y_hat_AR2, data_list$y, model_order = 2)
> mse_AR3 <- calculate_mse(y_hat_AR3, data_list$y, model_order = 3)
> # Calculate total log-likelihood for each chain
> total_log_lik2 <- function(log_lik) {
+   apply(log_lik, 1, sum)  # Sum all log-likelihood values
+ }
> log_lik_AR1_total2 <- mean(total_log_lik2(log_lik_AR1))
> log_lik_AR2_total2 <- mean(total_log_lik2(log_lik_AR2))
> log_lik_AR3_total2 <- mean(total_log_lik2(log_lik_AR3))
> # Create a data frame to store the results
> ICs_df <- data.frame(
+   Model = c("AR(1)", "AR(2)", "AR(3)"),
+   LOOIC = c(loo_AR1_total$estimates["looic", "Estimate"], loo_AR2_total$estimates["looic", "Estimate"], loo_AR3_total$estimates["looic", "Estimate"]),
+   p_eff_LOOIC = c(loo_AR1_total$estimates["p_loo", "Estimate"], loo_AR2_total$estimates["p_loo", "Estimate"], loo_AR3_total$estimates["p_loo", "Estimate"]),
+   MSE = c(mse_AR1, mse_AR2, mse_AR3),
+   LL = c(log_lik_AR1_total2, log_lik_AR2_total2, log_lik_AR3_total2)
+ )
> # Calculate weights based on LOOIC
> inverse_looic <- 1 / ICs_df$LOOIC
> weights_looic <- inverse_looic / sum(inverse_looic)
> # Calculate weights based on MSE (using inverse because lower MSE is better)
> inverse_mse <- 1 / ICs_df$MSE
> weights_mse <- inverse_mse / sum(inverse_mse)
> # Calculate weights based on LL (directly because higher LL is better)
> weights_ll <- ICs_df$LL / sum(ICs_df$LL)
> # Add weights to the data frame
> ICs_df$Weight_LOOIC <- weights_looic
> ICs_df$Weight_MSE <- weights_mse
> ICs_df$Weight_LL <- weights_ll
> # Function to calculate ensemble predictions
> calculate_ensemble <- function(y_hat_list, weights) {
+   y_hat_ensemble <- array(0, dim = dim(y_hat_list[[1]]))
+   for (i in 1:length(weights)) {
+     y_hat_ensemble <- y_hat_ensemble + weights[i] * y_hat_list[[i]]
+   }
+   return(y_hat_ensemble)
+ }
> # Calculate ensemble predictions for each set of weights
> y_hat_list <- list(y_hat_AR1, y_hat_AR2, y_hat_AR3)
> y_hat_ensemble_looic <- calculate_ensemble(y_hat_list, weights_looic)
> y_hat_ensemble_mse <- calculate_ensemble(y_hat_list, weights_mse)
> y_hat_ensemble_ll <- calculate_ensemble(y_hat_list, weights_ll)
> # Calculate MSE for each ensemble
> mse_ensemble_looic <- calculate_mse(y_hat_ensemble_looic, data_list$y, model_order = 3)
> mse_ensemble_mse <- calculate_mse(y_hat_ensemble_mse, data_list$y, model_order = 3)
> mse_ensemble_ll <- calculate_mse(y_hat_ensemble_ll, data_list$y, model_order = 3)
> # Add ensemble MSEs to the data frame
> ensemble_mse_df <- data.frame(
+   Ensemble = c("LOOIC", "MSE", "LL"),
+   MSE = c(mse_ensemble_looic, mse_ensemble_mse, mse_ensemble_ll)
+ )
\end{Sinput}
\end{Schunk}


\section{Results and Model Comparison}

\begin{Schunk}
\begin{Sinput}
> # Print the results in the desired format
> print(ICs_df)
\end{Sinput}
\begin{Soutput}
  Model    LOOIC p_eff_LOOIC      MSE        LL Weight_LOOIC Weight_MSE
1 AR(1) 1576.735    2.339213 2.849206 -787.0631    0.3042586  0.2606197
2 AR(2) 1401.491    3.797516 2.068039 -698.6273    0.3423035  0.3590644
3 AR(3) 1357.340    5.210934 1.952480 -675.7976    0.3534378  0.3803158
  Weight_LL
1 0.3641302
2 0.3232159
3 0.3126539
\end{Soutput}
\begin{Sinput}
> print(ensemble_mse_df)
\end{Sinput}
\begin{Soutput}
  Ensemble      MSE
1    LOOIC 1.405046
2      MSE 1.389718
3       LL 1.441284
\end{Soutput}
\end{Schunk}

\section{Model Diagnostics and Evaluation}
\begin{Schunk}
\begin{Sinput}
> # Function to calculate mean squared error (MSE) at each time point starting from the 3rd data point
> calculate_mse_time <- function(y_hat, y_true, start_point = 4) {
+   mse_values <- sapply(start_point:dim(y_hat)[3], function(t) {  # Iterate over time points starting from 'start_point'
+     mse_per_person <- sapply(1:length(y_true), function(i) (y_hat[1, i, t] - y_true[[i]][t])^2)
+     return(mean(mse_per_person)) # Average MSE across individuals for this time point
+   })
+   return(mse_values) # MSE for each time point
+ }
> # Calculate MSE for each model at each time point starting from the 3rd data point
> mse_time_AR1 <- calculate_mse_time(y_hat_AR1, data_list$y)
> mse_time_AR2 <- calculate_mse_time(y_hat_AR2, data_list$y)
> mse_time_AR3 <- calculate_mse_time(y_hat_AR3, data_list$y)
> # Calculate MSE for each ensemble at each time point starting from the 3rd data point
> mse_time_ensemble_looic <- calculate_mse_time(y_hat_ensemble_looic, data_list$y)
> # Define x-axis for the plot (starting from the 3rd data point)
> x_axis <- 4:n_timepoints
> # Prepare data for plotting
> mse_time_data <- data.frame(
+   Time = rep(x_axis, 4),
+   MSE = c(mse_time_AR1, mse_time_AR2, mse_time_AR3, mse_time_ensemble_looic),
+   Model = factor(rep(c("AR(1)", "AR(2)", "AR(3)", "Ensemble (LOOIC)"), each = length(x_axis)))
+ )
> # Plot MSE over time for each model and ensemble with y-axis on log scale
> ggplot(mse_time_data, aes(x = Time, y = MSE, color = Model, group = Model)) +
+   geom_line(size = 1.2) +
+   scale_y_log10() +
+   labs(title = "MSE over Time (Starting from 3rd Data Point)", x = "Time", y = "Log(MSE)") +
+   theme_minimal() +
+   theme(
+     plot.title = element_text(hjust = 0.5),
+     legend.position = "top",
+     legend.title = element_blank()
+   ) +
+   scale_color_manual(values = c("AR(1)" = "gray", "AR(2)" = "gray", "AR(3)" = "gray", "Ensemble (LOOIC)" = "purple"))
> # Prepare data for density plot
> error_data <- data.frame(
+   Error = c(mse_time_AR1, mse_time_AR2, mse_time_AR3, mse_time_ensemble_looic),
+   Model = factor(rep(c("AR(1)", "AR(2)", "AR(3)", "Ensemble (LOOIC)"), each = length(mse_time_AR1)))
+ )
> # Plot error distribution as density plot
> ggplot(error_data, aes(x = Error, color = Model, fill = Model)) +
+   geom_density(alpha = 0.4) +
+   scale_x_log10() +
+   labs(title = "Error Distribution by Model", x = "Log(Error)", y = "Density") +
+   theme_minimal() +
+   theme(legend.position = "top")
\end{Sinput}
\end{Schunk}

\end{document}
